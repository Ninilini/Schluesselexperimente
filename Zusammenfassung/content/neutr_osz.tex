\section{Neutrino Oscillations \cite{oszis}}
Neutrino oscillations are a direct consequence of the fact that neutrinos have masses. Several experiments had observed neutrino deficits for the solar and atmospheric neutrinos. The Super-Kamiokande and the Sudbury-Neutrino-Observatory made it possible to prove that these are caused by neutrino oscillations.
\subsection{Historical Context}
In 1957 Pontecorvo proposed neutrino oscillations for the first time. In 1962, the same year as the muon neutrino discovery, the PMNS matrix was introduced. It provided a framework to explain possible neutrino oscillations.\\
Since the neutrino cross section is of the order of $\sigma_{\nu} \sim 10^{-14} \si{\barn\per\giga\electronvolt}$ and a predictable flux of neutrinos is needed for experiments with neutrinos, the most suitable ones are solar and atmospheric neutrinos.
The solar neutrinos are only electron neutrinos that are produced by nuclear fusion in the pp chain reaction. The Standard Solar Model (SSM) by Bahcall made it possible to predict the steady flux. The $\beta^+$ decay of the $^{8}B$ provides neutrinos with the highest energy of up to \SI{18.8}{\mega\electronvolt}, but the flux is also the lowest.
The first experiment investigating the solar neutrino flux was the Homestake experiment. It was developed in the 1960s by R.Davis and placed in the Homestake mine \SI{1478}{\meter} underground. The detector consisted of a tank of perchloroethylene, that is able to detect neutrinos with an energy above \SI{0.814}{\mega\electronvolt}. The Argon produced by the reaction $\nu_e + ^{37}C \rightarrow e^- + ^{37}Ar$ was extracted and its decays counted. This lead to the discovry that only about one-third of the predicted SSM neutrinos reached the earth. When a cosmic ray hits the atmosphere it creates a shower that contains pions. The pions decay into muons which in turn decay into electrons. The expected ratio of the atmospheric neutrinos is $\frac{\nu_{\mu}}{\nu_e} \approx 2$. The Kamiokande experiment was build to search for the proton decay. Muon neutrinos are the main background. In 1988 the Super-Kamiokande experiment published their result of the ratio $\frac{(\nu_{\mu}/\nu_e)_\text{exp}}{(\nu_{\mu}/\nu_e)_\text{theo}} = 0.60 \pm 0.05$,
which indicates a deficit in atmospheric neutrinos. This result was later confirmed by the IMB experiment.

\subsection{Theoretical principles}
Assuming that neutrinos carry a mass, their flavour eigenstates ($\nu_i$ with $i=e,\mu,\tau$) are different from the mass eigenstates ($\nu_j$ with $j=1,2,3$). The PMNS (Pontecrovo-Maki-Nakagawa-Sakata) matrix $U$ describes the mixing of the mass eigenstates into the flavour eigenstates. $U$ is a $3\times3$ mixing matrix containing three real mixing angles and one complex phase factor.\\
The propagation of a mass eigenstate can be described by $\ket{\nu_j(t)} = e^{-iHt}\ket{\nu_j(0)} = e^{-iEt}\ket{\nu_j(0)}$, where the Hamiltonian operator is equivalent to the initial energy carried by the neutrino. It is the solution of the time-dependent Schrödinger equation described by the Hamiltonian operator in vacuum. The flavour eigenstate is then given by $\ket{\nu_i(t)} = \sum_i U^*_{ij}e^{-iH_it}\ket{\nu_j(0)}$. \\
In the simplified scenario of two \textcolor{red}{neutrino times}, the oscillation probabilty between two flavour eigentstates is given by:
\begin{align*}
	P(\nu_{F1}\rightarrow\nu_{F2}) = |\langle \nu_{F2} | \nu_{F1} \rangle|^2 = \sin^2 2\vartheta \sin^2 \frac{\Delta m^2 L}{4E} \;.
\end{align*}
Here $\vartheta$ is the mixing angle that affects the amplitude and $\Delta m^2 = m_2^2 - m_1^2$, which affects the frequency. Current measurements discovered that the mixing in the PMNS matrix is nearly maximal. Moreover, only the mass differences, but not the masses themselves, could be determined so far. They are $|\Delta m_{21}^2| = 7.55 ^{+0.20}_{-0.16}10^{-5}\si{\electronvolt}^2$ and $|\Delta m_{31}^2| = 2.50 ^{+0.03}_{-0.03}10^{-3}\si{\electronvolt}^2$.

\subsection{Super-Kamiokande}
The motivation for the neutrino flux measurement of Super-Kamiokande was the fact that neutrinos can easily traverse the earth due to their small cross section. As shown before, the oscillation probability depends on the distance the neutrinos travel. The upward and downward going neurtinos cover significantly different distances. So if the observed anomaly is due to oscillation, the flavour composition of the neutrino flux should depend on the angle.\\
Super-Kamiokande is a cylindrical Cherenkov Detector that contains \SI{22500}{\tonne} of ultra pure water and \num{11146} Photomultipliers. The neutrinos interact weakly with a nucleus and produce a charged lepton. Electron events are smeared out compared to the muon events due to bremsstrahlung effects. In 1998 the collaboration published the first significant evidence for atmospheric neutrino oscillations. They reported a significant zenith dependent deficit of the $\mu$-like events compared to the probability of no oscillations. The channels of the $\nu_e$ and the downward going $\nu_{\mu}$ events were in agreement with the expactation of no oscillation. This behavior has been seen in the sub-\si{\giga\electronvolt} and multi-\si{\giga\electronvolt} sector. The ratio $R=\frac{(\mu/e)_{\text{Data}}}{(\mu/e)_{\text{MC}}}$ was $R=0.63 \pm 0.05 \; (\text{sub-}\si{\giga\electronvolt}) \; 0.65\pm 0.08 \; (\text{multi-}\si{\giga\electronvolt})$.
The data have been consistent with the two flavour $\nu_{\mu},\nu_{\tau}$ oscillation.

\subsection{The Sudbury Neutrino Observatory}
The Sudbury Neutrino Observatory (SNO) was build to investigate the nature of the solar neutrino deficit. Possible sources could be neutrino oscillations, a neutrino decay or that the SSM is incorrect. In contrast to previous experiments the SNO was not only sensitive to $\nu_e$, but also the sum of all neutrino types. This is due to the use of Deuterium as detector material. The SNO was a Cherenkov detector, that was placed \SI{2}{\kilo\meter} underground and contained over 9000 photomultipliers.\\
The neutrinos interact in three different ways:
\begin{itemize}
	\item charged current (CC) interaction: $d + \nu_e \rightarrow \ p + p + e^-$.
  The neutrino energies reach up to \SI{15}{\MeV}, but the mass of the muon is much greater. This is the reason why the charged current reactions are just possible for electron neutrinos. ($\Phi_{\text{CC}} = \Phi(\nu_e)$)
  \textcolor{red}{detektierbar}.
  \item neutral current (NC) interaction: $d + \nu_x \rightarrow \ \nu_x + n + p$.
  This process is sensitive to all types of neutrinos. The neutron capture leads to a photon cascade, which can interact via compton or photo-effect and produce high-energetic electrons. These produce Cherenkov light with an isotropic distribution. Hence it is possible to seperate NC and CC events. ($\Phi_{\text{NC}} = \Phi(\nu_e) + \Phi(\nu_{\mu/\tau})$)

  \item electron elastic scattering (ES): $\nu_x + e^- \rightarrow \ \nu_x + e^-$
  The elastic scattering with a valence electron is far more senstitive to electron neutrinos than to other types. It is strongly correlated to the neutrino direction. ($\Phi_{\text{ES}}= \Phi(\nu_e) + 0.1559\Phi(\nu_{\mu/\tau})$)
\end{itemize}
The angular distribution showed a slight anticorrelation for the CC, a strong correlation for the ES and no correlation at all for the NC to the angle of the neutrinos. The measured fluxes are
\begin{align*}
	\Phi_{\text{CC}} &= 1.76^{+0.06}_{-0.05}(\text{stat.})^{+0.09}_{-0.09}(\text{syst.})\\
	\Phi_{\text{NC}} &= 2.39^{+0.24}_{-0.23}(\text{stat.})^{+0.12}_{-0.12}(\text{syst.})\\
	\Phi_{\text{ES}} &= 5.09^{+0.44}_{-0.43}(\text{stat.})^{+0.46}_{-0.43}(\text{syst.})
\end{align*}
All fluxes counted together fit the predicted flux of the SSM. Together with the measurement of a significant flux of $\nu_{\mu/\tau}$, this was the confirmation that the solar neutrino deficit is due to neutrino oscillations.\\
The discovery of neutrino oscillations was an important step in neutrino physics. It proved that neutrinos have a mass and opened the question of the mass hierachy and the absolute masses of the neutrinos. Other unsolved problems are the nature of the neutrino masses, the $CP$ violation in the leptonic sector and the search for sterile right-handed neutrinos. These problems will be investigated with future experiments like KATRIN (masses), Hyper-Kamiokande ($CP$ violation) and SNO+ (mass nature).
