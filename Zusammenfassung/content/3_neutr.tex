\section{The Discovery of Three Types of Neutrinos \cite{neutrino}}
The discovery of all three types of neutrinos took nearly 50 years. It was again a success of the Standard model, which for example predicted the $\nu_{\tau}$ 25 years before it was finally discovered in 2000.

\subsection{Fermi's Theory}
In 1930 Pauli predicted the neutrino to explain the observed electron energy spectrum of the $\beta$ decay. In 1934 Fermi developed his theory of the $\beta$ decay based on Dirac's quantum field theory of electromagnetism. Since the weak force has a short range, he proposed a point like interaction to transfer the electric and weak charge between the two currents. These are build out of the two interacting particles of the initial and final state. The coupling strength $G_F$ is valid for all processes of this form, since they are based on the same interaction. The crossing symmetry allows processes to be rearranged. The $\beta$ decay of the neutron $n \rightarrow p + e^- + \bar{\nu}_e$ can therefore be transfromed to the inverse $\beta$ decay $p + \bar{\nu}_e \rightarrow n + e^+$. This process was important for the detection of neutrinos.\\
Today we know that a $W^{\pm}$ boson is exchanged between the two currents in the $\beta$ decay. The short range of the weak interaction is a result of massive gauge bosons. The third boson $Z^0$ of the weak interaction does not change the flavour of the current as the $W^{\pm}$ do. Fermi's theory did not predict the bosons and is therfore just valid on short distances. This manifests itself in the mass-dimension of the coupling strength $G_F$.

\subsection{Discovery of the electron neutrino}
The principle of spin conservation lead to the conclusion that the neutrino is a fermion with spin $\frac{1}{2}$. It was already known from the discovery of the positron and the antiproton, that all fermions have corresponding antiparticles. Since there was no measurement of a mass or a magnetic moment, it remained unclear whether the neutrino is a Dirac or Majorana particle. The latter would mean that the neutrino was its own antiparticle. The nature of the neutrino is under investigation by e.g. the search for the neutrinoless double $\beta$ decay. If the neutrino is a Majorana particle it will reabsorb itself and the neutrinos would vanish in the decay $^{150}Nd \; \rightarrow \; ^{150}Sm + 2 e^{-} + 2 \bar{\nu}_e$.\\
The cross section of neutrinos could be calculated from the $\beta$ energy spectrum and $G_F$. The problem was that $\sigma_{\nu} \propto 10^{-44}$, leading to an infinite matter penetrability for low energy neutrinos, where their energy is smaller as the reaction threshold. The penetrability is still large for \si{\mega\electronvolt} neutrinos. The first idea was to use the inverse $\beta$ decay and detect the positrons and neutrons, that are produced. A large fusion reactor could provide a high rate of antineutrinos, but a reasonable signal to background rate is needed, to discover the neutrino.\\
The first attempt took place at the Hanford experiment with a plutonium-producing fission reactor in 1953. It contianed a 300-liter liquid scintillator to provide the protons. Since the mean free path of the neutron is longer as the positron one, the time delay between two signals was measured. The delay was estimated to be approximately \SI{9}{\micro\second} to be an antineutrino signal. But a significant signal to background rate was not measured, because the influence of cosmic rays was underestimated.\\
The Savannah River experiment finally lead to the discovery of the electron neutrino. It was placed deep underground and surrounded by a lead-paraffin shielding to minimize the influence of cosmic rays. It consisted of three tanks with a liquid scintillator inside and photomultipliers at the ends. Between these three scintillator tanks two tanks were placed, that contained a target solution made of water and cadmium chlorid. If a neutrino entered the tank and hit a proton the positron and neutron are produced. The positron annihilates nearly immediately with an electron under the emission of two \SI{0.511}{\mega\electronvolt} photons. Due to its longer free path the neutron remains longer until it is captured by the cadmium under the emission of a \SI{9}{\mega\electronvolt} photon. The three detector tanks were connected to one input of a triple-beam oscilloscope. Two pulse heights with the corresponding energy of the two processes and a time delay of \SI{5.5}{\micro\second} needed to be measured to be counted as a signal. Due to the geometric of the experiment the signal also showed up in two read-out signals. This lead to a signal to background ratio of 3 to 1 and was awarded in 1995 with the Nobel prize as the discovery of the electron (anti-)neutrino.

\subsection{Discovery of the Muon Neutrino}
Under the assumption of two different kinds of neutrinos that only produce the corresponding lepton, e.g. $\nu_{e}/\bar{\nu}_e \rightarrow e^{-}/e^{+}$ the charged pion decays into one lepton and its corresponding neutrino. Since the electron mass is much smaller than the muon mass, the decay into an electron and $\nu_{e}$ is helicity suppressed.
Pontecorvo and Schwartz independently proposed to use particle accelerators to search for the muon neutrino. The Brookhaven Alternating Gradient Synchrotron (AGS) had an alternating magnetic field to focus the proton beam - even at high energies. The resulting \SI{15}{\giga\electronvolt} beam was collided with a beryllium target on a straight section. This produced pions with a small angle respectively to the synchroton ring, that afterwards mainly decayed into muons and neutrinos. The spark chamber detector was shielded with steel from the main background, strongly interacting particles and muons. The detector consisted of 10 one-ton modules, that were made out of 9 aluminium plates seperated by lucite spacers. The shielding made it impossible to produce neutron events. The collaboration observed 24 single muon events, where 5 got excluded, because they did not originate from the AGS. They could have been cosmic rays. If only one type of neutrino had existed the collaboration would also have had to detect 29 electron showers. Instead they just observed 6 possible electron showers. This lead to the discovery of the muon neutrino in 1962.

\subsection{Discovery of the Tau Neutrino}
After the discovery of the $\tau$ and $\nu_{\mu}$, the $\tau$ neutrino was predicted as the third neutrino generation. The $\tau$ neutrino was discovered by the DONUT collaboration in 2000. They used the collision of a \SI{800}{\giga\electronvolt} proton beam with a target to produce $D_s$ mesons, that are heavy enough to decay into $\tau$ leptons and $\nu_{\tau}$ neutrinos. \textcolor{red}{The decay into the $\nu_{\tau}$ is helicity suppressed compared to the $\tau$}. In order to ensure that just neutrinos pass through, there was a shielding placed between the target and the detector.
The detector is consisted of three iron blocks which were separated with a plastic layer between two emulsion layers. The principle that lead to the discovery of the $\tau$ neutrino was that if a $\nu_{\tau}$ collides with an iron nucleus, it produces a $\tau$ lepton which decays afterwards. By detecting the $\tau$ lepton with the emulsion layers, the collaboration indirectly proved the existence of the $\tau$ neutrino.
\subsection{Number of light neutrinos}
The $Z$ production at $e^+e^-$ colliders allows to precisely measure the number of light neutrinos. By subtracting the visible partial decay width $\Gamma_{\text{vis}}$ of the $Z$ boson into charged quarks and leptons from the total $Z$ decay width the partial decay with $\Gamma_{\text{inv}}$ into neutrinos remains. This can be used to calculate the number of neutrinos $N_{\nu}$ as following:
\begin{align*}
	N_{\nu} &= \frac{\Gamma_{\text{inv}}}{\Gamma_{l}} \left( \frac{\Gamma_l}{\Gamma_{\nu}}\right)_{\text{SM}} & \left( \frac{\Gamma_l}{\Gamma_{\nu}}\right)_{\text{SM}} = 1.991 \pm 0.001 \; .
\end{align*}
The results of all four LEP experiments lead to $N_{\nu} = 2.984 \pm 0.008$. Therefore the $\nu_{e}$, $\nu_{\mu}$ and $\nu_{\tau}$ are all light neutrino generations compatible with the $Z$ decay width.
